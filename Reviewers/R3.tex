% Reviewer 3
\reviewer

\begin{revcomment}
	Compared with previous works, the current work further considers the complex overlapping range relationship between sensors. Therefore, the authors should spend a lot of effort to emphasize this contribution in Section I.
\end{revcomment}
\begin{revresponse}
	Thanks for the valuable suggestion.
	We have revised Section 1 to better highlight the challenge introduced by this overlapping range relationship between sensors, and to underscore their practical relevance in real-world scenarios.

	The revised description of this relationship:
	\begin{changes}
		This altitude-dependent coverage creates intricate spatial relationships between sensors, manifesting in various overlapping patterns: partial overlap, full containment, reverse containment, and disjointness.
		Moreover, these relationships become even more complex as communication range changes with altitude, creating challenges that have not been fully addressed in previous research.
		For example, in Fig.~1, as the altitude increases, sensor $S_1$ initially overlaps with $S_2$, but eventually becomes fully contained by $S_2$.
		Such overlapping range relationship better captures the complexity of real-world scenarios and provides practical value for UAV-assisted data collection.
		Capturing and fully exploiting these altitude-dependent ranges is crucial, since it enables the data from a single sensor to be collected across multiple sessions and fundamentally influences the order of data collection.
	\end{changes}
	The third challenge in the revised manuscript:
	\begin{changes}
		Due to the complex overlapping patterns of sensor data transmission ranges, the UAV is often simultaneously located within the ranges of multiple sensors. Consequently, determining the optimal data collection sequence becomes highly challenging.
	\end{changes}
\end{revresponse}

\begin{revcomment}
	The system model assumes an idealized scenario without incorporating real-world path loss effects (e.g., fading, shadowing). This limits the practical use of the proposed algorithm.
\end{revcomment}
\begin{revresponse}
	Thank you for your comment.
	We have added Subsection 3.1.2 ``Communication Model'', where the probabilistic line-of-sight model is employed to model the communication between the UAV and sensors.
	The revised manuscript is shown below.
	\begin{changes}
		The UAV collects data from sensors via sensor-to-UAV links.
		In this work, the probabilistic LoS channel model is employed~\cite{a-constspeed2}
		The channel coefficient $H_i$ between the UAV and sensor $s_i$ at time $t$ can be represented as follows:
		\begin{equation}
			H_i(t) = \sqrt{\beta_i(t)}\widetilde{H}_i(t), 
		\end{equation}
		where $\widetilde{H}_i(t)$ denotes the complex coefficient of small-scale fading with $E[|\widetilde{H}_i(t)|^2]=1$, and $\beta_i(t)$ is the coefficient for the shadowing and path loss.
		Generally, $\beta_i(t)$ depends on LoS or non-LoS link condition~\cite{Zeng},
		\begin{equation}
		\beta_i(t)=\begin{cases}
			\widetilde{\beta}/d_i^\gamma(t), & \mbox{LoS link,} \\
			\kappa\widetilde{\beta}/d_i^\gamma(t), & \mbox{non-LoS link,} 
		\end{cases}
		\end{equation}
		where $\widetilde{\beta}$ reflects the path loss at unit distance, $d_i(t)$ represents the distance between the UAV and $s_i$, $\gamma$ is the path loss exponent, and $\kappa\in(0,1)$ captures the attenuation effect under the non-LoS conditions.

		Let $P^{LoS}_i(t)$ be the LoS probability between the UAV and sensor $s_i$.
		The expectation of the channel power gain is given as follows~\cite{a-constspeed2}:
		\begin{equation}
			E[|H_i(t)|^2] = P^{LoS}_i(t)\frac{\widetilde{\beta}}{d_i^{\gamma}(t)} + (1-P^{LoS}_i(t))\frac{\kappa\widetilde{\beta}}{d_i^{\gamma}(t)}.
		\end{equation}
		Finally, the average transmission rate between the UAV and sensor $s_i$ at time $t$ can be given as~\cite{trans-rate}:
		\begin{equation}
			R_i(t)=B\log_2(1+\frac{GE[|H_i(t)|^2]p_i}{N_0B}),
			\label{eq:rate}
		\end{equation}
		where $B$ is the bandwidth of the channel, $G$ denotes the antenna gain, $p_i$ denotes the transmission power, and $N_0$ represents the noise variance.
	\end{changes}
\end{revresponse}

\begin{revcomment}
	The claim of ``near-optimal performance'' for SSF-ACO-Online lacks a clear benchmark (e.g., theoretical bounds or exhaustive search). Provide a quantitative comparison to support this conclusion.
\end{revcomment}
\begin{revresponse}
	We appreciate the reviewer's insightful comment.
	We recognize that the term ``near-optimal'' was not precise. Our intention was to indicate that SSF-ACO-Online performs closely to the offline algorithm SSF-ACO. We have revised the abstract to clarify this point and removed the term ``near-optimal'' to avoid confusion.
	\begin{changes}
		Extensive simulations demonstrate that SSF-ACO significantly outperforms baseline approaches in energy efficiency, and SSF-ACO-Online achieves comparable performance with energy consumption 1.24\% higher than offline counterpart in average.
	\end{changes}
\end{revresponse}

\begin{revcomment}
	The robustness of the SSF-ACO and SSF-ACO-Online algorithms to network dynamics (e.g., node mobility) is not evaluated. Include tests under time-varying conditions to demonstrate adaptability.
\end{revcomment}
\begin{revresponse}
	We appreciate the reviewer's comment on the network dynamics.
	The ground nodes (GNs) mobility, failures, and the computational latency of SSF-ACO-Online are discussed in the following.

	\textbf{GN mobility.}
	GN mobility is an important factor in various UAV-assisted systems.
	However, our current work focuses on data collection from stationary sensors.
	In the scenarios we consider, the sensors are fixed in the environment to monitor infrastructures or natural conditions.
	Admittedly, a number of existing studies~\cite{GNmob1, GNmob2, GNmob3, GNmob4} have investigated GN mobility.
	In these studies, GNs typically refer to mobile user devices with significant movement, such as user-carried devices~\cite{GNmob1,GNmob2} or vehicles~\cite{GNmob3,GNmob4}, which are fundamentally different from the stationary GNs considered in our work.

	\textbf{Failures.}
	Failures may occur either before or during data transmission.
	Failures before transmission will result in the sensor being undetectable by the UAV.
	On the other hand, failures during the transmission process will lead to a disruption in the connection between the UAV and the sensor.
	The UAV will attemp to reconnect and, after reaching the maximum number of reconnection attempts, will abandon the connection.
	Regardless of whether reconnection is successful, the scheduling will be re-planned to ensure energy efficiency.

	\textbf{The computational latency of SSF-ACO-Online.}
	We have reported the cumulative computational time of SSF-ACO-Online and the corresponding UAV flight durations.
	These results are summarized in Table~\ref{tb:runtime2} (Table 4 in the manuscript) and discussed in the revised manuscript as follows to demonstrate the suitability of SSF-ACO-Online.

	\begin{table}[h]
		\renewcommand{\arraystretch}{1.2}
		\centering
		\caption{Computation Time of SSF-ACO and SSF-ACO-Online Compared with UAV Flight Duration in Online Scheduling (in Seconds)}
		\label{tb:runtime2}
		\centering
		\begin{tabular}{*{9}{c}}
			\hline
			$n$ & 5 & 10 & 15 & 20 & 25 & 30 & 35 & 40 \\
			\hline
			$T_f$ & 513.73 & 1107.11 & 1502.68 & 2084.67 & 2675.47 & 2997.83 & 3519.72 & 4119.79 \\
			$T_{off}$ & 1.58 & 5.60 & 12.01 & 19.11 & 32.32 & 51.01 & 64.60 & 92.73 \\
			$T_{on}$ & 0.99 & 1.81 & 2.91 & 3.09 & 3.49 & 3.87 & 5.04 & 5.80 \\
			$T_{on} / T_f$ & 0.19\% & 0.16\% & 0.19\% & 0.15\% & 0.13\% & 0.13\% & 0.14\% & 0.14\% \\
			\hline
		\end{tabular}
	\end{table}

	\begin{changes}
		Finally, we analyze the computational latency of SSF-ACO-Online.
		Table~4 reports the duration of UAV flights $T_f$ under online problem, along with the computation time of SSF-ACO-Online $T_{on}$, and that of SSF-ACO $T_{off}$ in the corresponding offline problem.
		Across varying numbers of sensors, the computation times of SSF-ACO-Online stay under 6 seconds and constitute no more than 0.19\% of the corresponding UAV flight duration.
		This demonstrates the suitability of SSF-ACO-Online for online applications.
		Since the active sensor list only maintains those sensors that have been discovered but not yet completely collected, though invoked SSF-ACO multiple times, SSF-ACO-Online has a cumulative computation time significantly shorter than that of SSF-ACO.
	\end{changes}
\end{revresponse}

\begin{revcomment}
	The paper uses multiple important symbols, but a comprehensive table of symbols and definitions is missing. Adding such a table would enhance readability and help readers locate information more easily.
\end{revcomment}
\begin{revresponse}
	Thanks you for the suggestion. We have added more important symbols and definitions in Table 1 in the manuscript to make it more comprehensive. The revised version is presented in Table~\ref{tb:notation} below.
	% \begin{changes}
	\begin{table}[h]
		\caption{Major notations}
		\label{tb:notation}
		\centering
		\begin{tabular}{c>{\raggedright\arraybackslash}p{10cm}}
			\toprule
			Notation  & \multicolumn{1}{c}{Explanation} \\
			\midrule
			$D$       & The destination point. \\
			$s_i$     & $i$-th sensor. \\
			$h_j$     & $j$-th altitude. \\
			$l_i, l_{i,h}$  & Entering boundary point of $s_i$. \\
			$r_i, r_{i,h}$  & Exiting boundary point of $s_i$. \\
			$R_i$     & Transmission rate of $s_i$. \\
			$t_i$     & Required transmission duration of $s_i$. \\
			$\mathbb{S}$    & Data structure including all sensor information. \\
			$f(d)$    & The time when UAV reaching position $d$. \\
			$v, v(d)$ & Speed of UAV at position $d$. \\
			$g(d)$    & Altitude of UAV at position $d$. \\
			$p(v)$    & UAV power consumption at forward speed $v$. \\
			$E_f$     & Horizontal energy consumption of UAV. \\
			$E_v$     & Vertical energy consumption of UAV. \\
			$E$       & Total energy consumption of UAV. \\
			$b_k$     & $k$-th boundary point. \\
			$C_k$     & $k$-th cell, $C_k=(b_k,b_{k+1})$. \\
			$U_S(i)$  & $=0$, sensor $s_i$ is inactive; $=1$, $s_i$ is active. \\
			$U_C(k)$  & $=0$, cell $C_k$ is unavailable; $=1$, $C_k$ is available. \\
			$S[k,k']$ & $=(b_k,b_{k'})$, path interval between $b_k$ and $b_{k'}$. \\
			$D[k,k']$ & Effective distance of $S[k,k']$. \\
			$T[k,k']$ & Effective time of $S[k,k']$. \\
			\bottomrule
		\end{tabular}
	\end{table}
	% \end{changes}
\end{revresponse}

\begin{revcomment}
	The discussion of related work is not sufficient. Please include recent literature on ``the fusion of AI, UAV, and LAE'' and provide further discussion.
	% todo 搜关键词
\end{revcomment}
\begin{revresponse}
	Thanks for the comment.
	We have strengthened our manuscript in two places.
	\begin{itemize}
		\item \textbf{Section 1}: We added a summary of UAV applications in the LAE to further elucidate the indispensable role of UAVs.
		\item \textbf{Section 2.3}: We expanded the discussion of AI techniques applied to UAV-assisted systems.
	\end{itemize}
	The revised passages in Section 1 and Section 2.3 are shown below.
	\begin{changes}
		UAVs, with their superior maneuverability, rapid deployment capabilities, and adaptive flight patterns, are indispensable to the LAE.
		They have been deployed for a wide range of applications, including last-mile package delivery~\cite{b-logistics}, simultaneous localization and mapping~\cite{SLAM}, and forestry~\cite{b-agriculture,none-river}.
		When equipped with onboard processing units, UAVs can serve as mobile servers, providing computational services to ground user or vehicles~\cite{a-constspeed2,IoV,b-DL,liao2024energy}, which supply temporary computing support and alleviating computational loads.
	\end{changes}
	\begin{changes}
		Learning-based methods, particularly deep learning (DL) and RL, gained popularity in recent years, as they learned complex parameters or action policies during the training process.
		In UAV-assisted mobile edge computing systems, Lin \etal~\cite{b-DL} proposed a parametrized dueling deep Q-network to maximize the UAV's energy efficiency.
		The problem was formulated as a mixed-integer nonlinear programming (MINLP) problem, which facilitated the modeling of problems involving both continuous variables (\eg trajectory) and discrete variables (\eg data collection decision and task offloading decision).
		To maximize the total throughput and energy efficiency, Chen \etal~\cite{b-RL} formulated long-term UAV-aided data collection problem as a Markov Decision Process (MDP), and addressed it by a multi-agent DRL algorithm.
		However, the maximum number of ground nodes that could be served in \cite{b-DL} and \cite{b-RL} was only 6 and 8, respectively, as the extremely high computational complexity became unacceptable when the number of ground nodes increased.
		Hao \etal~\cite{hao-minlp} formulated a task offloading problem in a UAV-assisted mobile edge computing system as a MINLP, and then transformed it into a MDP.
		The problem was solved by a DRL algorithm.
		Zhong \etal~\cite{zhong-minlp} also applied a similar approach to address a task offloading and resource allocation problem.
		Despite their successful application to larger-scale problems, challenges still remain in the context of the JUSAS problem, particularly in modeling the environment and designing the reward function, including aligning the reward signal with the objective and dealing with sparse rewards.
	\end{changes}
\end{revresponse}

\printpartbibliography{GNmob1,GNmob2,GNmob3,GNmob4,b-DL,b-RL,hao-minlp,zhong-minlp,b-logistics,SLAM,IoV,b-agriculture,none-river,a-constspeed2,liao2024energy,Zeng,trans-rate}
