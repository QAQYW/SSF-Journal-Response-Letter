% Reviewer 3
\reviewer

\begin{revcomment}
	Compared with previous works, the current work further considers the complex overlapping range relationship between sensors. Therefore, the authors should spend a lot of effort to emphasize this contribution in Section I.
\end{revcomment}
\begin{revresponse}
	Thanks for the valuable suggestion.
	We have revised Section 1 to better highlight the challenge introduced by this overlapping range relationship between sensors, and to underscore their practical relevance in real-world scenarios.

	The revised description of this relationship:
	\begin{changes}
		This altitude-dependent coverage creates intricate spatial relationships between sensors, manifesting in various overlapping patterns: partial overlap, full containment, reverse containment, and disjointness.
		Moreover, these relationships become even more complex as communication range changes with altitude, creating challenges that have not been fully addressed in previous research.
		For example, in Fig.~1, as the altitude increases, sensor $S_1$ initially overlaps with $S_2$, but eventually becomes fully contained by $S_2$.
		Such overlapping range relationship better captures the complexity of real-world scenarios and provides practical value for UAV-assisted data collection.
		Capturing and fully exploiting these altitude-dependent ranges is crucial, since it enables the data from a single sensor to be collected across multiple sessions and fundamentally influences the order of data collection.
	\end{changes}
	The third challenge in the revised manuscript:
	\begin{changes}
		Due to the complex overlapping patterns of sensor data transmission ranges, the UAV is often simultaneously located within the ranges of multiple sensors. Consequently, determining the optimal data collection sequence becomes highly challenging.
	\end{changes}
\end{revresponse}

\begin{revcomment}
	The system model assumes an idealized scenario without incorporating real-world path loss effects (e.g., fading, shadowing). This limits the practical use of the proposed algorithm.
\end{revcomment}
\begin{revresponse}
	% todo System Model
\end{revresponse}

\begin{revcomment}
	The claim of ``near-optimal performance'' for SSF-ACO-Online lacks a clear benchmark (e.g., theoretical bounds or exhaustive search). Provide a quantitative comparison to support this conclusion.
\end{revcomment}
\begin{revresponse}
	We appreciate the reviewer's insightful comment.
	We recognize that the term ``near-optimal'' was not precise. Our intention was to indicate that SSF-ACO-Online performs closely to the offline algorithm SSF-ACO. We have revised the abstract to clarify this point and removed the term ``near-optimal'' to avoid confusion.
	\begin{changes}
		Extensive simulations demonstrate that SSF-ACO significantly outperforms baseline approaches in energy efficiency, and SSF-ACO-Online achieves comparable performance with energy consumption 1.24\% higher than offline counterpart in average.
	\end{changes}
\end{revresponse}

\begin{revcomment}
	The robustness of the SSF-ACO and SSF-ACO-Online algorithms to network dynamics (e.g., node mobility) is not evaluated. Include tests under time-varying conditions to demonstrate adaptability.
\end{revcomment}
\begin{revresponse}
	We appreciate the reviewer's comment on the network dynamics.
	The ground nodes (GNs) mobility, failures, and the computational latency of SSF-ACO-Online are discussed in the following.

	\textbf{GN mobility.}
	GN mobility is an important factor in various UAV-assisted systems.
	However, our current work focuses on data collection from stationary sensors.
	In the scenarios we consider, the sensors are fixed in the environment to monitor infrastructures or natural conditions.
	Admittedly, a number of existing studies~\cite{GNmob1, GNmob2, GNmob3, GNmob4} have investigated GN mobility.
	In these studies, GNs typically refer to mobile user devices with significant movement, such as user-carried devices~\cite{GNmob1,GNmob2} or vehicles~\cite{GNmob3,GNmob4}, which are fundamentally different from the stationary GNs considered in our work.

	\textbf{Failures.}
	Failures may occur either before or during data transmission.
	Failures before transmission will result in the sensor being undetectable by the UAV.
	On the other hand, failures during the transmission process will lead to a disruption in the connection between the UAV and the sensor.
	The UAV will attemp to reconnect and, after reaching the maximum number of reconnection attempts, will abandon the connection.
	Regardless of whether reconnection is successful, the scheduling will be re-planned to ensure energy efficiency.

	\textcolor{red}{\textbf{The computational latency of SSF-ACO-Online.}}
	% todo complete
\end{revresponse}

\begin{revcomment}
	The paper uses multiple important symbols, but a comprehensive table of symbols and definitions is missing. Adding such a table would enhance readability and help readers locate information more easily.
\end{revcomment}
\begin{revresponse}
	\textcolor{red}{
	Thanks you for the suggestion. We have added more important symbols and definitions in Table 1 to make it more comprehensive. The revised Table 1 is as follows.
	}
	\begin{changes}
		% todo 完善Table 1后复制到这
	\end{changes}
\end{revresponse}

\begin{revcomment}
	The discussion of related work is not sufficient. Please include recent literature on ``the fusion of AI, UAV, and LAE'' and provide further discussion.
	% todo 搜关键词
\end{revcomment}
\begin{revresponse}
	% todo Introduction
\end{revresponse}

\printpartbibliography{GNmob1,GNmob2,GNmob3,GNmob4}
