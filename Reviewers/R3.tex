% Reviewer 3
\reviewer

\begin{revcomment}
	Compared with previous works, the current work further considers the complex overlapping range relationship between sensors. Therefore, the authors should spend a lot of effort to emphasize this contribution in Section \Romannum{1}.
\end{revcomment}
\begin{revresponse}
	% Yes. When using \href{https://www.ctan.org/pkg/biblatex}{biblatex}, you can use the \texttt{refsection=section} option to achieve that.
	% If we cite a new reference like \cite{Besser2021} here, it will again be number [1].
	
	% Note that you might have to run \texttt{pdflatex} and \texttt{biber} multiple times.
	
	% And reference [1] for Reviewer 2~\cite{ReviewerReference} is now number [2].
	
	% \printpartbibliography{Besser2021,ReviewerReference}
\end{revresponse}

\begin{revcomment}
	The system model assumes an idealized scenario without incorporating real-world path loss effects (e.g., fading, shadowing). This limits the practical use of the proposed algorithm.
\end{revcomment}
\begin{revresponse}
	
\end{revresponse}

\begin{revcomment}
	The claim of ``near-optimal performance'' for SSF-ACO-Online lacks a clear benchmark (e.g., theoretical bounds or exhaustive search). Provide a quantitative comparison to support this conclusion.
\end{revcomment}
\begin{revresponse}
	We appreciate the reviewer's insightful comment.
	We agree that the term ``near-optimal'' was not precise. Our intention was to indicate that SSF-ACO-Online performs closely to the offline algorithm SSF-ACO. We have revised the abstract to clarify this point and removed the term ``near-optimal'' to avoid confusion.
	\begin{changes}
		Extensive simulations demonstrate that SSF-ACO significantly outperforms baseline approaches in energy efficiency, and SSF-ACO-Online achieves comparable performance with energy consumption 1.24\% higher than offline counterpart in average.
	\end{changes}
\end{revresponse}

\begin{revcomment}
	The robustness of the SSF-ACO and SSF-ACO-Online algorithms to network dynamics (e.g., node mobility) is not evaluated. Include tests under time-varying conditions to demonstrate adaptability.
\end{revcomment}
\begin{revresponse}
	
\end{revresponse}

\begin{revcomment}
	The paper uses multiple important symbols, but a comprehensive table of symbols and definitions is missing. Adding such a table would enhance readability and help readers locate information more easily.
\end{revcomment}
\begin{revresponse}
	
\end{revresponse}

\begin{revcomment}
	The discussion of related work is not sufficient. Please include recent literature on ``the fusion of AI, UAV, and LAE'' and provide further discussion.
\end{revcomment}
\begin{revresponse}
	
\end{revresponse}

