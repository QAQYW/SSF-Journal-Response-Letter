% Reviewer 2
\reviewer

\begin{revcomment} % \label{cmt:work-not-good}
	The computational latency of SSF-ACO-Online needs justify.
\end{revcomment}
\begin{revresponse}
	Thank you for the comment.
	% todo 运行时间的实验数据
\end{revresponse}

\begin{revcomment}
	Alg. 4 does not specify the specific implementation of ``Roulette-Wheel-Selection'', which needs to supplement pseudocode or reference standard methods.
\end{revcomment}
\begin{revresponse}
	Thanks for your suggestion.
	The classic Roulette wheel selection method is applied, and we have added a brief description of the Roulette-Wheel-Selection method in the revised manuscript.
	\begin{changes}
		Specifically, in \textbf{Roulette-Wheel-Selection}, the vertex $(\varphi+1, j')$ is selected if $\sum_{j''=0}^{j'-1}{p(\varphi,j,j'')}\leq \text{rand}() < \sum_{j''=0}^{j'}{p(\varphi,j,j'')}$, where $\text{rand}()$ generates a random number in $[0,1)$.
	\end{changes}
\end{revresponse}

\begin{revcomment}
	The vertical energy consumption is assumed to be linear related to the altitude difference, which seems oversimplified, more justification is needed.
\end{revcomment}
\begin{revresponse}
	Thanks for the comment.
	Research~\cite{vertical-assumption} serves as the theoretical basis for this ``linear assumption''.
	According to~\cite{vertical-assumption}, the following equation fits well with the theoretical derivation,
	\begin{equation}
		p_V = c_{0,V} + c_{1,V}v + c_{2,V}v^2,
	\end{equation}
	where $p_V$ is the vertical power consumption, $c_{0,V}$, $c_{1,V}$ and $c_{2,V}$ are coefficients, and $v$ is the climbing or descending speed of the UAV.
	In our manuscript, we assume that the UAV climbs or descends at a constant speed and flies in a stable environment.
	Therefore, the vertical power consumption is constant.
	For a certain altitude difference $\Delta h$, the vertical energy consumption is $E_V=\frac{p_V\Delta h}{v}$, which supports the ``linear assumption''.
	Furthermore, similar assumptions have also been adopted in previous research~\cite{mgh}.
\end{revresponse}

\begin{revcomment}
	The difference between the transmission range model of Equation (17) and literature [35] is not fully explained, and the improvement points or advantages need to be clearly defined.
\end{revcomment}
\begin{revresponse}
	
\end{revresponse}

\begin{revcomment}
	What is the purpose of presenting Equation (18)?
\end{revcomment}
\begin{revresponse}
	Thans you for the comment.
	Equation (18) in the original manuscript was included to provide a quantitative description of the data transmission range's width.
	To retain a more holistic presentation, we have modified the expression.
	\begin{changes}
		The width and height of the range are linearly related to the coefficients $C_W$ and $C_H$, respectively.
		And the quantitave relationship can be found in Appendix F of the supplementary material.
	\end{changes}
\end{revresponse}

\begin{revcomment}
	There are too many curves in Figure 8 and the colors are similar, so it is difficult to distinguish them. It is suggested to optimize the color matching or add mark symbols
\end{revcomment}
\begin{revresponse}
	Thanks for your suggestion.
	We have revised the color scheme and marker symbols to make the curves in Figure 8 in our manuscript, and also adjusted the layout of it to improve clarity.
	% todo 新图复制过来?
\end{revresponse}

\begin{revcomment}
	The dynamic nature of the sensor is not considered, such as movement, inaccurate position, failure and other problems.
\end{revcomment}
\begin{revresponse}
	
\end{revresponse}

\begin{revcomment}
	Some minor suggestions:\\
	(1) The the first sentence in subsection 2.1, ``UAV'' → ``UAVs''.\\
	(2) Second paragraph of subsection 2.1, ``maintaining'' → ``maintained''.\\
	(3) Second paragraph of subsection 2.2, ``maintaining'' → ``and maintain''.\\
	(4) The ``='' in table 1.\\
	(5) Add $k \neq k'$ to equation (11), and $i \neq i'$ to equation (12).\\
	(6) The parameter $\mathbb{S}$ of algorithm 1 and 2 is neither used nor updated.\\
	(7) Line 2 of Algorithm 3, why set 20 to $Z$?
\end{revcomment}
\begin{revresponse}
	% todo 要不要把对应部分复制过来
	Thanks for your detailed suggestions.
	In response, we have made the following revisions.

	For points (1) to (4), we have improved the phrasing and table in the manuscript to mamke it more consistent and easier to read.

	For point (5), we would like to clarify that in Definition 1 in the manuscript, the condition $k<k'$ is defined. Additionally, in Equation (11), the condition $b_k < b_{k'}$ already implicitly ensures that $k\neq k'$.
	As for Equation (12), we have removed it along with its related content to improve the overall structure and clarity of the manuscript.

	For point (6), $\mathbb{S}$ is defined as a data structure that encapsulates essential sensor information. To improve clarity and facilitate readers' understanding, we have revised the manuscript by relocating the description of $\mathbb{S}$ closer to Algorithm 1 and 2.
	\begin{changes}
		For details, Alg. 1 takes the necessary sensor information as input and returns the optimal horizontal energy consumption.
		For simplicity, we denote these pieces of sensor information (such as $l_i$ and $r_i$) by $\mathbb{S}$.
	\end{changes}

	For point (7), we have added an explanation in the revised manuscript.
	\begin{changes}
		Here, $Z$ is set to 20 to balance solution quality and computational efficiency.
	\end{changes}
\end{revresponse}

\printpartbibliography{vertical-assumption,mgh}
